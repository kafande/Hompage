% Created 2018-08-20 lun 19:12
% Intended LaTeX compiler: pdflatex
\documentclass[12pt,letterpaper]{article}
\usepackage[sans]{preamble}
\author{Carlos E S}
\date{August 21, 2018}
\title{Recitation Session 1}
\hypersetup{
 pdfauthor={Carlos E S},
 pdftitle={Recitation Session 1},
 pdfkeywords={},
 pdfsubject={},
 pdfcreator={Emacs 26.1 (Org mode 9.1.13)}, 
 pdflang={English}}
\begin{document}

\maketitle
\tableofcontents


\section{Last Quiz Statistics}
\label{sec:orgbe3074b}
Since we have not yet had a quiz, this section is irrelevant. But in the
future, I will be putting quiz statistics (i.e., average and standard
deviation) here.

\section{Most Missed Problems}
\label{sec:orge582558}
Unlike previous terms, I will not be providing full solutions to the quizzes;
instead I will provide solutions to the problems my sections struggled with
the most. This will be less of a burden on me and, I hope, will encourage you
to attend recitation to get your questions answered.

\section{Recitation Notes}
\label{sec:org068df55}
Since this will be the first recitation, we will mostly be talking about
ground rules and course policies for the semester. 

First off, I have a website, it's \url{https://math.purdue.edu/\~salinac/ma261-f18}.
There you will find these notes as well as useful links and blank quizzes.
Unfortunately, my office hours have not been finalized as of yet; this is due
to the fact that the Math Help Room (or the Math Resource Room, as it is now
known) has gone through some major changes, and the administration has been
delayed in assigning office hours. At any rate, once I know what my office
hours for the semester will be, I will make sure to let you know both in
person and in writing.

Now, let's get down to some base rules. 

\subsection{Quizzes}
\label{sec:org97c786a}
There will be quizzes every recitation, excluding this one. The material will
resemble that of your previous lecture. To get a more precise description,
please look at the course calendar. 


\subsection{Late Homework and Missed Quizzes}
\label{sec:org52bce11}
Late homework will not be accepted and no make-up quizzes will be given. At
the end of the semester the three lowest homework and three lowest quizzes
are dropped. However, I can make excuses on an individual basis; especially
if you missed that homework assignment or that quiz due to medical issues. 

\subsection{Midterms}
\label{sec:org8029e8d}
There will be two multiple-choice one-hour midterms These are on
\begin{itemize}
\item Midterm 1 Tuesday, October 2, 8:00 PM, at Elliott Hall of Music
\item Midterm 2 Wednesday, October 31, 6:30 PM at Elliot Hall of Music
\end{itemize}

Notice that both of these two midterms are not scheduled at the same time; I
will make a note of this again, as the date approaches.

\subsection{Grades}
\label{sec:org06c1682}
Course grades will be determined in roughly the following way:
\begin{center}
\begin{tabular}{lr}
Homework & 100\\
Quizzes & 50\\
Midterm 1 & 100\\
Midterm 2 & 100\\
Final & 200\\
\hline
\textbf{Total} & 550\\
\end{tabular}
\end{center}

The cutoffs for certain letter grades will be about the following: 300
A+/A/A-, 320 B+/B/B-, 280 C+/C/C-, 240 D+/D.
\end{document}